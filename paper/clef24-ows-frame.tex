%% The first command in your LaTeX source must be the \documentclass command.
%%
%% Options:
%% twocolumn : Two column layout.
%% hf: enable header and footer.
\documentclass[
% twocolumn,
% hf,
]{ceurart}

%%
%% One can fix some overfulls
\sloppy

%%
%% Minted listings support 
%% Need pygment <http://pygments.org/> <http://pypi.python.org/pypi/Pygments>
\usepackage{listings}
%% auto break lines
\lstset{breaklines=true}

%%
%% end of the preamble, start of the body of the document source.
\begin{document}


%%
%% Rights management information.
%% CC-BY is default license.
\copyrightyear{2024}
\copyrightclause{Copyright for this paper by its authors. Use permitted under Creative Commons License Attribution 4.0 International (CC BY 4.0).}

%%
%% This command is for the conference information
\conference{CLEF'24: Conference and Labs of the Evaluation Forum, September 09--12, 2024, Grenoble, France}

\title{Team OpenWebSearch at CLEF 2024: LongEval and QuantumCLEF}

%%
%% The "author" command and its associated commands are used to define
%% the authors and their affiliations.
\address[1]{Radboud Universiteit Nijmegen}
\address[2]{Friedrich-Schiller-Universit{\"a}t Jena}
\address[3]{University of Kassel, hessian.AI, ScaDS.AI}

%alphabetical order, with first group PhD and second group PI
\author[1]{Daria Alexander}[]
\author[2]{Maik Fr{\"o}be}[]
\author[1]{Gijs Hendriksen}[]
\author[2]{Ferdinand Schlatt}[]
\author[2]{Matthias Hagen}[]
\author[1]{Djoerd Hiemstra}[]
\author[3]{Martin Potthast}[]
\author[1]{Arjen P. de Vries}[]


%%
%% The abstract is a short summary of the work to be presented in the
%% article.
\begin{abstract}
  Quantum annealers can better explore complex optimization landscapes than . the problem formulation that we build is, due to the bootstrapping, highly inconsistent/contradictory. Hence, we explore quantom annealers, comparing them with simulated annealing, finding that xy. QuantumCLEF: Feature Selection for Bootstrapped Feature Importance.
  LongEval incorporate historical data.
\end{abstract}

%%
%% Keywords. The author(s) should pick words that accurately describe
%% the work being presented. Separate the keywords with commas.
\begin{keywords}
  LaTeX class \sep
  paper template \sep
  paper formatting \sep
  CEUR-WS
\end{keywords}

%%
%% This command processes the author and affiliation and title
%% information and builds the first part of the formatted document.
\maketitle

\section{Introduction}

CEUR-WS's article template provides a consistent \LaTeX{} style for
use across CEUR-WS publications, and incorporates accessibility and
metadata-extraction functionality. This document will explain the
major features of the document class.

If you are new to publishing with CEUR-WS, this document is a valuable
guide to the process of preparing your work for publication.

\section{Related Work}

Bootstrapping for feature selection was already used before~\cite{yuan:99}, but not in IR/LTR.


Hello world example~\cite{yuan:99}: Add forgery features that are not relevant to the pool. Try to reduce them. I cann add those features in any distribution, normal distribution, etc. E.g., I should be able to fool every feature selection that does not incorporate the true relevance label.


\section{Selecting Important Features with QUBO} 

QUBO = quadratic unconstrained binary optimization minimizes~\cite{pasin:2024}:

\begin{equation*}
\vec{x}^{T} \cdot Q \cdot \vec{x} = \sum_{i}^{N} q_{i} \cdot x_{i} + \sum_{i<j}^N q_{i,j} \cdot x_{i} \cdot x_{j}
\end{equation*}



\section{Conclusion}
\label{conclusion}

TBD...

\begin{acknowledgments}
  This publication has received funding from the European Union's Horizon Europe research and innovation programme under grant agreement No 101070014 (OpenWebSearch.EU, \url{https://doi.org/10.3030/101070014})
\end{acknowledgments}


%%
%% Define the bibliography file to be used
\bibliography{sample-ceur}

%%
%% If your work has an appendix, this is the place to put it.
\appendix

\section{Online Resources}


The sources for the ceur-art style are available via
\begin{itemize}
\item \href{https://github.com/yamadharma/ceurart}{GitHub},
% \item \href{https://www.overleaf.com/project/5e76702c4acae70001d3bc87}{Overleaf},
\item
  \href{https://www.overleaf.com/latex/templates/template-for-submissions-to-ceur-workshop-proceedings-ceur-ws-dot-org/pkfscdkgkhcq}{Overleaf
    template}.
\end{itemize}

\end{document}

%%
%% End of file
